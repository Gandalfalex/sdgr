% !TeX program = xelatex
\chapter{Fragestellung}

Aufbauend auf den bereits existierenden Kenntnissen \dots TODO kurze Conclusion zum State of the Art Teil \dots

% Recherche und Analyse bestehender Literatur und aktuellen Möglichkeiten, um Daten basierend auf existierenden Datensätzen zu erzeugen, mit besonderem Augenmerk auf ML und TSD/TSA



\section{Anforderungsermittlung für eine Software, die Daten auf Grundlage eines bestehenden Datensatzes simuliert}

Die Software richtet sich an Personen, die mit einem speziellen Vorwissen im Bereich der Signalverarbeitung ausgestattet sind.
Aber es handlet sich nicht um Experten und nicht zwangsläufig um Mathematiker. Die Hauptzielgruppe besteht Hauptsächlich aus Ingeneuren.

Aufbauend auf dieser Annahme muss folgende Punkte besonders geachtet werden:

\begin{enumerate}
    \item Einfache und Effiziente UI 
    \item Konfigurationsmöglichkeiten der Daten 
    \item Wiederholbarkeit der Versuche
    \item Persistenz der Daten und Ergebnisse
    \item Effizienz der Generierung
\end{enumerate}



\subsection{Einfache und Effiziente UI}

Nicht zwangsläufig Experten Nutzer -> Sollten Begrifflichkeiten kennen bzw den Sinn dahinter verstehen, aber werden nicht ständig mit der Software arbeiten 
-> Schritte müssen einfach und Nachvollziehbar sein
-> muss ohne großen Aufwand wieder 'erlernbar sein'
-> themen sind relativ abstrackt, viel visuelle Unterstützung bieten

Schneidermans 8 golden Rules of User Interface Design mit einbringen

\subsection{Konfigurationsmöglichkeiten der Daten}
Dies ist an 2 Stellen möglich, einerseits in Ursprünglicher Software über datasets und dann nochmal in den jeweiligen ML/TSD implmentationen

Einfachheit, Erklärung der jeweiligen Komponenten, eventuell deren Einfluss angeben (Nutzer nicht zwangsläufig im Bereich ML/TSD bewandert)


\subsection{Wiederholbarkeit der Versuche}
Versuche oder Simulationen müssen einsehbar und Konstant sein. Kein Mehrwert wenn Abweichung zwischen den Simulationen zu groß ist


\subsection{Persistenz der Daten und Ergebnisse}
Es handelt sich um eine Webanwendung, teilweise computational demanding 

Modelle müssen gespeichert werden, wiederaufrufbar sein
TestDaten müssen gespeichert werden und über mehrere Modelle zugänglich
-> Daten müssen an Nutzer gebunden sein (Ideal wäre vermutlich Nutzergruppe..? Hier könnte man diskutieren)



\section{Konzeption einer Lösung, um Daten mittels Maschine Learning zu generieren, um so Sensoren zu simulieren}
