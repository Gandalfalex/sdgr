\chapter{Personas}
\label{cha:Personas}
Personas sind fiktionale Repräsentationen von Zielgruppen oder Nutzern, die deren Charakteristika, Verhalten und Bedürfnisse aufzeigen. 
Im \acl{UCD} dienen sie dazu, benutzerorientierte Lösungen zu entwickeln. Da sich Personas an den realen Nutzern orientieren, helfen sie Designern und Entwicklern, 
einen tieferen Einblick in ihre Zielgruppe zu gewinnen. Der dadurch entstehende Designprozess ist nutzerorientiert und ermöglicht die Bereitstellung von Lösungen, 
die dem Kunden oder Nutzer einen echten Mehrwert bieten. Des Weiteren erleichtern Personas die Kommunikation innerhalb von Teams\cite{Personas96:online}, indem sie als gemeinsamer Bezugspunkt 
dienen, Verwirrungen beseitigen und die Bemühungen aller Beteiligten aufeinander abstimmen. Oft wird im Erstellungsprozess deutlich, welche Punkte für den Nutzer besonders wichtig sind, was zu einer Priorisierungsreihenfolge führen kann.

\section{Fathima Olsson}
\subsection*{Hintergrund}
Fathima ist eine junge Medizintechnik-Studentin, die unter anderem im Bereich der Diabetes-Warngeräte forscht. Ihr Ziel ist es, die Fähigkeiten dieser Geräte zu erweitern. 
Da diese Geräte den Blutzuckerspiegel messen und den Nutzer warnen müssen, wenn dieser zu niedrig ist, muss das neue Gerät fehlerfrei funktionieren. Für die Validierung der 
Geräte sind umfangreiche Tests erforderlich, für die sie anonymisierte Testdaten mit Blutzuckerverläufen benötigt.

\subsection*{Demographie}
\personadetail{Geschlecht}{weiblich}
\personadetail{höchste Ausbildung}{Abitur}
\personadetail{Einkommen}{-}
\personadetail{Familienstand}{alleinstehend}
\personadetail{Sprachkenntnisse}{Deutsch, Englisch, Schwedisch}

\subsection*{Ziele}
\begin{itemize}
    \item Aufbau einer einfachen Testumgebung für schnelle und sichere Geräteentwicklung.
    \item Erweiterung der Fachkenntnisse im Bereich Qualitätskontrolle medizinischer Geräte.
\end{itemize}

\subsection*{Herausforderungen}
\begin{itemize}
    \item Beschaffung kostengünstiger und präziser Daten, die bedarfsgerecht angepasst oder erstellt werden können.
    \item Aufbau von Testszenarien, die auch kritische Werte simulieren können.
    \item Entwicklung eines sicheren und prüfbar funktionalen Konzepts zum Testen medizinischer Technologien.
\end{itemize}

\subsection*{Verhalten}
\begin{itemize}
    \item Forschungsorientierung: Fathima ist neugierig und motiviert, in ihrem Fachgebiet zu forschen und neue Erkenntnisse zu gewinnen.
    \item Präzision: Sie arbeitet genau und sorgfältig, insbesondere bei der Durchführung von Tests und Validierungen.
    \item Problemlösungskompetenz: Sie hat die Fähigkeit, komplexe Probleme zu analysieren und kreative Lösungen zu entwickeln.
    \item Zielorientierung: Ihr Ziel ist es, Menschen zu helfen, und sie ist bereit, die notwendigen Schritte zu unternehmen, um dies zu erreichen.
\end{itemize}

\subsection*{Bedürfnisse}
\begin{itemize}
    \item Einfache Nutzung: Das System muss einfach und flexibel einsetzbar sein.
    \item Nachnutzung eigener Daten: Die Software muss Möglichkeiten bieten, mit eigenen Daten zu arbeiten.
\end{itemize}

\subsection*{Zitat}
``Wenn es darum geht, Menschen in schwierigen Situationen zu helfen, darf man sich keine groben Fehler erlauben.``

\section{Dieter Maibach}
\subsection*{Hintergrund}
Dieter Maibach ist ein 50-jähriger Maschinenbau-Ingenieur, der vor etwa 30 Jahren studiert hat und seitdem in derselben Firma arbeitet. 
Er und seine Kollegen sind hauptsächlich für die Entwicklung neuer Mikrocontroller zuständig. Diese sind hochspezialisiert und müssen daher sehr stabil und zuverlässig laufen, 
weshalb sie eine umfangreiche Testsuite benötigen. Die Testsuite muss in der Lage sein, Signale, die von Sensoren an den Mikrocontroller gesendet werden, zu simulieren, 
um zu überprüfen, ob er richtig reagiert. Da das Ganze in einem großen Unternehmen stattfinden soll und frei zugänglich sein muss, muss die Anwendung frei zugänglich sein.

\subsection*{Demographie}
\personadetail{Geschlecht}{männlich}
\personadetail{höchste Ausbildung}{Hochschulabschluss}
\personadetail{Einkommen}{60.000}
\personadetail{Familienstand}{verheiratet}
\personadetail{Sprachkenntnisse}{Deutsch, Englisch (A1), Russisch}

\subsection*{Ziele}
\begin{itemize}
    \item Einführung von Qualitätsverbesserungsmethoden in den Entwicklungsprozess, um die Stabilität und Zuverlässigkeit der entwickelten Mikrocontroller zu steigern.
    \item Erweiterung der Fachkenntnisse im Bereich der Mikrocontroller-Entwicklung, um mit neuen Technologien und Designkonzepten auf dem neuesten Stand zu sein.
\end{itemize}

\subsection*{Herausforderungen}
\begin{itemize}
    \item Generierung von realistischen Testdaten und deren Konfiguration.
    \item Aufbau von Testszenarien, die auch kritische Werte simulieren können.
    \item Einführung von neuen Methoden/Tools in einen alten, lang bestehenden Entwicklungsprozess.
\end{itemize}

\subsection*{Verhalten}
\begin{itemize}
    \item Gewissenhaftigkeit: Dieter arbeitet präzise, genau und sorgfältig.
    \item Analytisches Denken: Er hat eine ausgeprägte Fähigkeit, komplexe Probleme zu analysieren und logische Lösungen zu finden.
    \item Innovationsfähigkeit: Dieter ist offen für neue Ideen und hat die Fähigkeit, kreative Lösungsansätze zu entwickeln.
    \item Beharrlichkeit: Er gibt nicht schnell auf und ist bereit, Herausforderungen anzunehmen und hartnäckig an Lösungen zu arbeiten.
    \item Teamorientierung: Dieter arbeitet gerne im Team und bringt einen kooperativen Ansatz ein.
\end{itemize}

\subsection*{Bedürfnisse}
\begin{itemize}
    \item Einfaches UI/UX Design: Die Software muss einfach zu bedienen und möglichst selbsterklärend sein.
    \item Einfache Bedienung: Die Software muss einfach in bestehende Umgebungen integrierbar sein.
    \item Flexible Nachnutzung: Testszenarien müssen im Team einfach nachnutzbar sein.
\end{itemize}

\subsection*{Zitat}
``Solange es zuverlässig und einfach ist, kann ich mich gerne mit neuen Sachen anfreunden.``
